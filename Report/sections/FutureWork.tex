\subsection{Calibration}
The TART interferometer needs to have a calibration step applied to it before it can be used to create images, otherwise the images it creates will not be completely correct. The calibration step can be done by creating a source with a strong radio signal nearby, for instance on top of a mountain near Stellenbosch that the interferometer can detect a strong signal from. This signal can then be used as a base for calibrating the interferometer as there is a source with a known signal strength.\\
Another method of calibration would be to use the positions of GPS satellites as sources with known signal strength and location\cite{CALIBRATION_TART}. This would work in a similar way to the previous method and is known as self calibration. 
\subsection{Deconvolution}
The TART interferometer needs a step after the imaging process that reduces the noise that is in the image. This noise comes in the form of the point spread function. The deconvolution step will reduce and perhaps remove the point spread function from the image. This will allow only the actual sources that are detected to be displayed and the image will be much clearer to the user.
\subsection{Transform the pipeline into a teaching tool}
This will allow the pipeline to teach people who are interested in interferometry about how it all works. 
The goal is to create a tool that can be used at a university level to teach students about interferometry in a way where they have access to a working interferometer and a pipeline that will explain every step in high detail to the student. Each step will explain the theory behind it and explain the process that was followed in that step to produce the results that are seen. Hopefully this will increase the ease of learning for interferometry and make more people interested in the field.